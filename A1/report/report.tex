\documentclass[a4paper, oneside]{article}
\special{pdf:minorversion 6}

\usepackage[english, russian]{babel}

\usepackage{fontspec}
\setmainfont[
  Ligatures=TeX,
  Extension=.otf,
  BoldFont=cmunbx,
  ItalicFont=cmunti,
  BoldItalicFont=cmunbi,
]{cmunrm}
\usepackage{unicode-math}

\usepackage[bookmarks=false]{hyperref}
\hypersetup{pdfstartview={FitH},
            colorlinks=true,
            linkcolor=magenta,
            pdfauthor={Павел Соболев}}

\usepackage{calrsfs}
\DeclareMathAlphabet{\pazocal}{OMS}{zplm}{m}{n}

\usepackage[table]{xcolor}
\usepackage{booktabs}
\usepackage{caption}

\usepackage{graphicx}
\graphicspath{ {../plots/} }

\usepackage{sectsty}
\sectionfont{\centering}
\subsubsectionfont{\centering\normalfont\itshape}

\newcommand{\npar}{\par\vspace{\baselineskip}}

\usepackage{float}
\input{../../shared/julia_font}
\input{../../shared/julia_listings}
\lstset{language=Julia, style=julia}

\setlength{\parindent}{0pt}


\hypersetup{pdftitle={Вычислительный практикум (9-ый семестр, 2021)}}

\begin{document}

\subsubsection*{Вычислительный практикум (9-ый семестр, 2021)}
\section*{Интегральные уравнения первого рода}
\subsubsection*{Руководитель: А. Г. Доронина \hspace{2em} Выполнил: П. Л. Соболев}

\vspace{3em}

\subsection*{Задачи}

\begin{itemize}
  \setlength\itemsep{-0.1em}
  \item Получить решение интегрального уравнения первого рода при заданных интервалах интегрирования, ядре и правой части.
\end{itemize}

\subsection*{Теория}

Интегральное уравнение первого рода:

\begin{equation}
  \int_a^b K(x, s) \, z(s) \, ds = u(x)
\end{equation}

Будем полагать, что ядро $ K(x, s) $ есть вещественная, непрерывная в области $ \{ a \leqslant s \leqslant b; \, c \leqslant x \leqslant d \} $ функция. Возьмем в качестве стабилизирующего функционала $ \Omega \, [z] $ функционал вида

\begin{equation}
  \Omega \, [z] = \int_a^b \{ z^2 + p(z')^2 \} \, ds,
\end{equation}

где $ p $ --- положительное число. \npar

Обозначим за $ F_n $ класс непрерывных на $ [a, b] $ функций $ z(s) $, имеющих обобщенные производные до $ n $-го порядка, интегрируемые с квадратом на $ [a, b] $. Пусть точное решение $ z_\textup{т}(s) $ принадлежит $ F_1 $ и удовлетворяет краевым условиям: $ z'(a) = 0 $, $ z'(b) = 0 $. Тогда в качестве регуляризованных решений $ z_{\alpha} $ уравнения (1) можно брать функции, являющиеся решениями следующей краевой задачи для уравнения Эйлера:

\begin{equation}
  \int_a^b \overline{K}(s, t) \, z(t) \, dt + \alpha \{ z(s) - p z''(s) \} = g(s),
\end{equation}

\begin{equation}
  z'(a) = 0, z'(b) = 0,
\end{equation}

где

\begin{equation}
  \overline{K}(s, t) = \int_c^d K(x, s) \, K(x, t) \, dx, \;\; g(s) = \int_c^d K(x, s) \, u(x) \, dx
\end{equation}

Напишем разностный аналог уравнения (3) на равномерной сетке с шагом $ h $. Разобьем промежуток $ [a, b] $ на $ n $ равных частей и возьмем в качестве узловых точек сетки середины полученных отрезков, т. е. полагаем

\begin{equation}
  s_i = a + 0.5 \cdot h + (i - 1) h, \quad i = 1, 2, \ldots, n; \quad h = \frac{b - a}{n}
\end{equation}

Заменив в левой части уравнения (3) интеграл соответствующей ему интегральной суммой, например, по формуле прямоугольников, а $ z''(s) $ --- соответствующим разностным отношением, получим

\begin{equation}
  \sum_{j=1}^n \overline{K}(s_i, t_j) \, h z_j + \alpha z_i + \alpha \, \frac{2 z_i - z_{i-1} - z_{i+1}}{h^2} = g_i,
\end{equation}

где

\begin{equation}
  i = 1, 2, \ldots, n, \quad g_i = \int_c^d K(x, s_i) \, u(x) \, dx
\end{equation}

Значения $ \overline{K}(s_i, t_j) $ и $ g_i $ либо вычисляются аналитически, либо получаются с помощью соответствующих квадратурных формул. \npar

При $ i = 1 $ и $ i = n $ в (7) входят не определенные ещё значения $ z_0 $ и $ z_{n+1} $. Чтобы удовлетворить граничным условиям, полагаем $ z_0 = z_1 $ и $ z_{n+1} = z_n $. Пусть $ B $ --- матрица с элементами $ B_{ij} = \overline{K}(s_i, t_j) \, h $. Тогда систему уравнений (7) относительно вектора $ z $ с компонентами $ (z_1, z_2, \ldots, z_n) $ можно записать в виде

\begin{equation}
  B_{\alpha} z \equiv Bz + \alpha C z = g,
\end{equation}

где $ g $ --- вектор с компонентами $ (g_1, g_2, \ldots, g_n) $, а $ \alpha C $ --- симметричная матрица вида

\[
\everymath{\displaystyle}
\begin{bmatrix}
  \alpha \left( 1 + \frac{1}{h^2} \right) & -\frac{\alpha}{h^2} & 0 & \dots & 0 & 0 \\
  -\frac{\alpha}{h^2} & \alpha \left( 1 + \frac{2}{h^2} \right) & -\frac{\alpha}{h^2} & \dots & 0 & 0 \\
  0 & -\frac{\alpha}{h^2} & \alpha \left( 1 + \frac{2}{h^2} \right) & \dots & 0 & 0 \\
  \vdots & \vdots & \vdots & \ddots & \vdots & \vdots \\
  0 & 0 & 0 & \dots & \alpha \left( 1 + \frac{2}{h^2} \right) & -\frac{\alpha}{h^2} \\
  0 & 0 & 0 & \dots & -\frac{\alpha}{h^2} & \alpha \left( 1 + \frac{1}{h^2} \right) \\
\end{bmatrix}
\]

Таким образом, задача сводится к решению СЛАУ (9).

\subsection*{Реализация}

Алгоритм реализован на языке программирования \href{https://julialang.org/}{Julia} в виде скрипта и расположен в GitLab репозитории \href{https://gitlab.com/paveloom-g/university/computational-workshop-s09-2021}{Computational Workshop S09-2021} в папке A1. Для воспроизведения результатов следуй инструкциям в файле {\footnotesize \texttt{README.md}}. \npar

Для проверки алгоритма возьмем интегральное уравнение

\begin{equation}
  \int_0^1 e^{sx} \, z(s) \, ds = \frac{e^{x+1} - 1}{x + 1},
\end{equation}

которое имеет аналитическое решение $ z(s) = e^z $.

\newpage

\captionsetup{singlelinecheck=false, justification=justified}

\begin{figure}[h!]
\begin{lstlisting}[
    caption=Определение ядра и правой части тестового уравнения
]
# Define the kernel (Test)
K(x, s) = exp(s * x)

# Define the right part of the equation (Test)
u(x) = (exp(x + 1) - 1) / (x + 1)
\end{lstlisting}
\end{figure}

\begin{figure}[h!]
\begin{lstlisting}[
  caption=Определение интервала и подготовка узлов
]
# Set the integration intervals
a = 0
b = 1
c = 0
d = 1

# Set the number of nodes
n = 100

# Set the initial value of the regularization parameter
α = 0.001

# Calculate the step
h = (b - a) / n

# Calculate the nodes for the s argument
s = [ a + 0.5 * h + (i - 1) * h for i in 1:n ]

# Calculate the nodes for the t argument
t = copy(s)
\end{lstlisting}
\end{figure}

\begin{figure}[h!]
\begin{lstlisting}[
  caption={Вычисление вектора $ g $, подготовка других переменных}
]
# Compute the g vector
g = [
  quadgk(x -> K(x, s[i]) * u(x), c, d; rtol=1e-8)[1]
  for i in 1:n
]

# Prepare a matrix for the computation
Bα = Matrix{Float64}(undef, n, n)

# Prepare a vector for the solution
z = Vector{Float64}(undef, n)

# Prepare a range of nodes for the residual calculation
xr = range(c, d; length=1000)
\end{lstlisting}
\end{figure}

\newpage

Функция {\footnotesize \texttt{quadgk}} (из пакета \href{https://juliamath.github.io/QuadGK.jl/stable/}{\footnotesize \texttt{QuadGK.jl}}) вычисляет значение интеграла методом Гаусса--Кронрода. \npar

Программа создает и оптимизирует функцию с параметром $ \alpha $, возвращающую значение невязки для вычисленного решения:

\begin{figure}[h!]
\begin{lstlisting}[
  caption={Определение функции, вычисляющей невязку}
]
# Compute the residual of the solution with the
# specified regularization parameter
function residual(θ::Vector{Float64})::Float64
    # Unpack the parameters
    α = θ[1]

    # Compute the Bα matrix
    for i in 1:n, j in 1:n
        αc = if (i == j == 1) || (i == j == n)
            α * (1 + 1 / h^2)
        elseif i == j
            α * (1 + 2 / h^2)
        elseif (i == j + 1) || (i + 1 == j)
            -α / h^2
        else
            0
        end
        Bα[i, j] =
        quadgk(
          x -> K(x, s[i]) * K(x, t[j]), c, d; rtol=1e-8
        )[1] * h + αc
    end

    # Compute the solution
    z .= Symmetric(Bα) \ g

    # Calculate the residual
    r = norm(
      [ sum(K.(x, s) .* z .* h) for x in xr ] .- u.(xr)
    )

    return r
end
\end{lstlisting}
\end{figure}

\newpage

\begin{figure}[h!]
\begin{lstlisting}[
  caption={Оптимизация по значению невязки}
]
# Optimize over the regularization parameter
res = Optim.optimize(
    residual,
    [α,],
    LBFGS(),
    Optim.Options(
        show_trace = false,
        extended_trace = true,
        store_trace = true,
    );
    inplace = false,
)
\end{lstlisting}
\end{figure}

Полученные значения параметра регуляризации и невязки для тестового интегрального уравнения: $ \alpha = 1.5813849 \cdot 10^{-7} $,
$r = 0.0021192 $. \npar

График сравнения решений:

\captionsetup{justification=centering}

\begin{figure}[h!]
  \centering
  \includegraphics[scale=0.5]{test}
  \caption{Сравнение точного и приближенного решений тестового интегрального уравнения}
\end{figure}

Ядро и правая часть из условий задания:

\begin{equation}
  K(x, s) = \frac{1}{1 + x + s}
\end{equation}

\begin{equation}
  u(x) = \frac{1}{\sqrt{2 - x}} \left( \ln{\frac{2 + x}{1 + x}} + 2 \ln{\frac{\sqrt{3} + \sqrt{2 - x}}{2 + \sqrt{2 - x}}} \right)
\end{equation}

Отрезки интегрирования те же: $ [0, 1] $.

\newpage

\captionsetup{justification=justified}

\begin{figure}[h!]
\begin{lstlisting}[
  caption={Определение ядра и правой части заданного уравнения}
]
# Define the kernel
K(x, s) = 1 / (1 + x + s)

# Define the right part of the equation
u(x) = 1 / sqrt(2 - x) * (log((2 + x) / (1 + x)) +
       2 * log(
         (sqrt(3) + sqrt(2 - x)) / (2 + sqrt(2 - x)))
       )
\end{lstlisting}
\end{figure}

Полученные значения параметра регуляризации и невязки для заданного интегрального уравнения: $ \alpha = 2.2101166 \cdot 10^{-6} $,
$r = 0.0007930 $. \npar

График решения:

\captionsetup{justification=centering}

\begin{figure}[h!]
  \centering
  \includegraphics[scale=0.5]{assignment}
  \caption{Приближенное решение заданного интегрального уравнения}
\end{figure}

\end{document}
