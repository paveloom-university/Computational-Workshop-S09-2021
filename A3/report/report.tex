\documentclass[a4paper, oneside]{article}
\special{pdf:minorversion 6}

\usepackage[english, russian]{babel}

\usepackage{fontspec}
\setmainfont[
  Ligatures=TeX,
  Extension=.otf,
  BoldFont=cmunbx,
  ItalicFont=cmunti,
  BoldItalicFont=cmunbi,
]{cmunrm}
\usepackage{unicode-math}

\usepackage[bookmarks=false]{hyperref}
\hypersetup{pdfstartview={FitH},
            colorlinks=true,
            linkcolor=magenta,
            pdfauthor={Павел Соболев}}

\usepackage{calrsfs}
\DeclareMathAlphabet{\pazocal}{OMS}{zplm}{m}{n}

\usepackage[table]{xcolor}
\usepackage{booktabs}
\usepackage{caption}

\usepackage{graphicx}
\graphicspath{ {../plots/} }

\usepackage{sectsty}
\sectionfont{\centering}
\subsubsectionfont{\centering\normalfont\itshape}

\newcommand{\npar}{\par\vspace{\baselineskip}}

\usepackage{float}
\input{../../shared/julia_font}
\input{../../shared/julia_listings}
\lstset{language=Julia, style=julia}

\setlength{\parindent}{0pt}


\hypersetup{pdftitle={Вычислительный практикум (9-ый семестр, 2021)}}

\lstset{basicstyle={\linespread{1.1}\JuliaMonoRegular\small\color[HTML]{19177C}}}

\begin{document}

\subsubsection*{Вычислительный практикум (9-ый семестр, 2021)}
\section*{Дифференциальные уравнения эллиптического типа}
\subsubsection*{Руководитель: А. Г. Доронина \hspace{2em} Выполнил: П. Л. Соболев}

\vspace{3em}

\subsection*{Задачи}

\begin{itemize}
  \setlength\itemsep{-0.1em}
  \item Создать алгоритм решения дифференциального уравнения эллиптического типа при заданном виде дифференциального оператора и протестировать его на предложенном решении.
\end{itemize}

\subsection*{Теория}

Постановка задачи: найти решение $ u(x, t) $ проблемы

\su
\begin{equation}
  -L u = f(x, t), \; (x, y) \in G,
\end{equation}

\su\su
\begin{equation}
  u = \mu (x, y), \; (x, y) \in \Gamma,
\end{equation}

где $ G \cup \Gamma = \left\{ 0 \leqslant x \leqslant l_x, \; 0 \leqslant y \leqslant l_y \right\} $, а

\su
\begin{equation}
  Lu = \frac{\partial}{\partial x} \left( p(x, y) \frac{\partial u}{\partial x} \right) + \frac{\partial}{\partial y} \left( q(x, y) \frac{\partial u}{\partial y} \right).
\end{equation}

Здесь $ p(x, y) $ и $ q(x, y) $ --- достаточно гладкие функции, такие, что \\ $ 0 \leq c_1 \leqslant p(x, y) \leqslant c_2 $, $ 0 \leq d_1 \leqslant q(x, y) \leqslant d_2 $, где $ c_1, c_2, d_1, d_2 $ --- постоянные. \npar

Разностный аналог дифференциального оператора (3) на равномерной сетке $ \overline{\omega}_{h_x h_y} = \left\{ (x_i = h_x i, y_j = h_y j), \; h_x = 1 / N, \; h_y = 1 / M, \; i = \overline{0, N}, \; j = \overline{0, M} \right\} $:

\begin{equation}
\begin{aligned}
  L_h u_{i, j} & = p_{i+\frac{1}{2}, j} \frac{u_{i+1, j} - u_{i, j}}{h_x^2} - p_{i-\frac{1}{2}, j} \frac{u_{i, j} - u_{i-1, j}}{h_x^2} + \\
  & + q_{i, j+\frac{1}{2}} \frac{u_{i, j+1} - u_{i, j}}{h_y^2} - q_{i, j-\frac{1}{2}} \frac{u_{i, j} - u_{i, j-1}}{h_y^2}
\end{aligned}
\end{equation}

Решение системы (1)--(2) сводится к решению линейной системы $ AU = F $ путем итераций ($ m $ -- оптимальное число итераций, $ \varepsilon $ -- требуемая точность) над внутренними узлами матрицы $ U $. В качестве нулевой итерации берется матрица $ U_0 $, такая, что её значения в граничных узлах ($ \gamma_{h_x h_y} $) совпадают со значениями функции $ \mu(x, y) $, а значения во внутренние узлах ($ \omega_{h_x h_y} $) равны нулю. \npar

Рассмотренные методы:

\begin{itemize}
  \item Метод простой итерации
  \begin{equation}
    u_{i, j}^k = \frac{\frac{p_{i-\frac{1}{2}, j} u_{i-1, j}^{k-1}}{h_x^2} + \frac{p_{i+\frac{1}{2}, j} u_{i+1, j}^{k-1}}{h_x^2} + \frac{q_{i, j-\frac{1}{2}} u_{i, j-1}^{k-1}}{h_y^2} + \frac{q_{i, j+\frac{1}{2}} u_{i, j+1}^{k-1}}{h_y^2} + f_{i, j}}{\frac{p_{i-\frac{1}{2}, j}}{h_x^2} + \frac{p_{i+\frac{1}{2}, j}}{h_x^2} + \frac{q_{i, j-\frac{1}{2}}}{h_y^2} + \frac{q_{i, j+\frac{1}{2}}}{h_y^2}},
  \end{equation}
  \begin{equation}
    \delta = c_1 \frac{4}{h_x^2} \sin^2{\frac{\pi h_x}{2 l_x}} + d_1 \frac{4}{h_y^2} \sin^2{\frac{\pi h_y}{2 l_y}},
  \end{equation}
  \begin{equation}
    \Delta = c_2 \frac{4}{h_x^2} \cos^2{\frac{\pi h_x}{2 l_x}} + d_2 \frac{4}{h_y^2} \cos^2{\frac{\pi h_y}{2 l_y}},
  \end{equation}
  \begin{equation}
    \xi = \delta / \Delta,
  \end{equation}
  \begin{equation}
    m \geqslant \frac{\ln(1 / \varepsilon)}{2 \xi}.
  \end{equation}
  \item Метод Зейделя
  \begin{equation}
    u_{i, j}^k = \frac{\frac{p_{i-\frac{1}{2}, j} u_{i-1, j}^k}{h_x^2} + \frac{p_{i+\frac{1}{2}, j} u_{i+1, j}^{k-1}}{h_x^2} + \frac{q_{i, j-\frac{1}{2}} u_{i, j-1}^k}{h_y^2} + \frac{q_{i, j+\frac{1}{2}} u_{i, j+1}^{k-1}}{h_y^2} + f_{i, j}}{\frac{p_{i-\frac{1}{2}, j}}{h_x^2} + \frac{p_{i+\frac{1}{2}, j}}{h_x^2} + \frac{q_{i, j-\frac{1}{2}}}{h_y^2} + \frac{q_{i, j+\frac{1}{2}}}{h_y^2}},
  \end{equation}
  \begin{equation}
    \delta = c_1 \frac{4}{h_x^2} \sin^2{\frac{\pi h_x}{2 l_x}} + d_1 \frac{4}{h_y^2} \sin^2{\frac{\pi h_y}{2 l_y}},
  \end{equation}
  \begin{equation}
    \Delta = c_2 \frac{4}{h_x^2} \cos^2{\frac{\pi h_x}{2 l_x}} + d_2 \frac{4}{h_y^2} \cos^2{\frac{\pi h_y}{2 l_y}},
  \end{equation}
  \begin{equation}
    \xi = \delta / \Delta,
  \end{equation}
  \begin{equation}
    m \geqslant \frac{\ln(1 / \varepsilon)}{4 \xi}.
  \end{equation}
  \item Метод верхней релаксации
  \begin{equation}
  \begin{aligned}
    u_{i, j}^k & = u_{i, j}^{k-1} + \\
    & + \omega \frac{p_{i+\frac{1}{2}, j} \frac{u_{i-1, j}^{k-1} - u_{i, j}^{k-1}}{h_x^2} - p_{i-\frac{1}{2}, j} \frac{u_{i, j}^{k-1} - u_{i-1, j}^k}{h_x^2}}{\frac{p_{i-\frac{1}{2}, j}}{h_x^2} + \frac{p_{i+\frac{1}{2}, j}}{h_x^2} + \frac{q_{i, j-\frac{1}{2}}}{h_y^2} + \frac{q_{i, j+\frac{1}{2}}}{h_y^2}} + \\
    & + \omega \frac{q_{i, j+\frac{1}{2}} \frac{u_{i, j+1}^{k-1} - u_{i, j}^{k-1}}{h_y^2} - q_{i, j-\frac{1}{2}} \frac{u_{i, j}^{k-1} - u_{i, j-1}^k}{h_y^2}}{\frac{p_{i-\frac{1}{2}, j}}{h_x^2} + \frac{p_{i+\frac{1}{2}, j}}{h_x^2} + \frac{q_{i, j-\frac{1}{2}}}{h_y^2} + \frac{q_{i, j+\frac{1}{2}}}{h_y^2}}, \\
    & \hspace{65pt} 0 \leq \omega \leq 2,
  \end{aligned}
  \end{equation}
  \begin{equation}
    \delta = c_1 \frac{4}{h_x^2} \sin^2{\frac{\pi h_x}{2 l_x}} + d_1 \frac{4}{h_y^2} \sin^2{\frac{\pi h_y}{2 l_y}},
  \end{equation}
  \begin{equation}
    \Delta = c_2 \frac{4}{h_x^2} \cos^2{\frac{\pi h_x}{2 l_x}} + d_2 \frac{4}{h_y^2} \cos^2{\frac{\pi h_y}{2 l_y}},
  \end{equation}
  \begin{equation}
    \xi = \delta / \Delta,
  \end{equation}
  \begin{equation}
    m \geqslant \frac{\ln(1 / \varepsilon)}{\sqrt{\xi}}.
  \end{equation}
  \item Попеременно-треугольный итерационный метод
  \begin{equation}
  \begin{gathered}
    \tilde{U}_{i, j}^k = \frac{\kappa_1 p_{i-\frac{1}{2}, j} \tilde{U}_{i-1, j} + \kappa_2 q_{i, j-\frac{1}{2}} \tilde{U}_{i, j-1} + L_h u_{i, j}^{k-1} + f_{i, j}}{1 + \kappa_1 p_{i-\frac{1}{2}, j} + \kappa_2 q_{i, j-\frac{1}{2}}}, \\
    i = 1, \ldots, N - 1, \; j = 1, \ldots, M - 1, \\
    \left. \tilde{U}^k \right|_{\gamma_{h_x h_y}} = 0,
  \end{gathered}
  \end{equation}
  \begin{equation}
  \begin{gathered}
    \overline{U}_{i, j}^k = \frac{\kappa_1 p_{i+\frac{1}{2}, j} \overline{U}_{i+1, j} + \kappa_2 q_{i, j+\frac{1}{2}} \overline{U}_{i, j+1} + \tilde{U}_{i, j}}{1 + \kappa_1 p_{i+\frac{1}{2}, j} + \kappa_2 q_{i, j+\frac{1}{2}}}, \\
    i = N - 1, \ldots, 1, \; j = M - 1, \ldots, 1, \\
    \left. \overline{U}^k \right|_{\gamma_{h_x h_y}} = 0,
  \end{gathered}
  \end{equation}
  \begin{equation}
    u_{i, j}^k = u_{i, j}^{k-1} + \tau \overline{U}_{i, j}^k,
  \end{equation}
  \begin{equation}
    \delta = c_1 \frac{4}{h_x^2} \sin^2{\frac{\pi h_x}{2 l_x}} + d_1 \frac{4}{h_y^2} \sin^2{\frac{\pi h_y}{2 l_y}}, \; \Delta = c_2 \frac{4}{h_x^2} + d_2 \frac{4}{h_y^2},
  \end{equation}
  \begin{equation}
    \eta = \delta / \Delta, \; \omega = 2 / \sqrt{\delta \Delta}, \; \gamma_1 = \frac{\delta}{2 + 2 \sqrt{\eta}}, \; \gamma_2 = \frac{\delta}{4 \sqrt{\eta}} \; \tau = 2 / (\gamma_1 + \gamma_2),
  \end{equation}
  \begin{equation}
    \kappa_1 = \omega / h_x^2, \; \kappa_2 = \omega / h_y^2, \; \xi = \gamma_1 / \gamma_2, \; \rho = (1 - \xi) / (1 + \xi),
  \end{equation}
  \begin{equation}
    m \geqslant \frac{\ln\varepsilon^{-1}}{\ln{\rho^{-1}}}.
  \end{equation}
  \item Метод переменных направлений
  \begin{equation}
  \begin{cases}
  \begin{alignedat}{3}
    & u_{0, j}^{k+\frac{1}{2}} && && = \mu(0, y_j), \\
    \overline{A}_{i, j} & u_{i-1, j}^{k+\frac{1}{2}} - \overline{B}_{i, j} u_{i, j}^{k+\frac{1}{2}} + \; \overline{C}_{i, j} && u_{i+1, j}^{k+\frac{1}{2}} && = \overline{G}_{i, j}^{k+\frac{1}{2}}, \; 1 \leqslant i \leqslant N - 1, \\
    & && u_{N, j}^{k+\frac{1}{2}} && = \mu(l_x, y_j),
  \end{alignedat}
  \end{cases}
  \end{equation}
  \begin{equation}
    \overline{G}_{i, j}^{k+\frac{1}{2}} = -u_{i, j}^k - \frac{\tau}{2} (\Lambda_2 u_{i, j}^k + f(x_i, y_j)), \; j = 1, \ldots, M - 1,
  \end{equation}
  \begin{equation}
  \begin{cases}
  \begin{alignedat}{3}
    & u_{i, 0}^{k+1} && && = \mu(x_i, 0), \\
    \overline{\overline{A}}_{i, j} & u_{i, j-1}^{k+1} - \overline{\overline{B}}_{i, j} u_{i, j}^{k+1} + \; \overline{\overline{C}}_{i, j} && u_{i, j+1}^{k+1} && = \overline{\overline{G}}_{i, j}^{k+1}, \; 1 \leqslant j \leqslant M - 1, \\
    & && u_{i, M}^{k+1} && = \mu(x_i, l_y),
  \end{alignedat}
  \end{cases}
  \end{equation}
  \begin{equation}
    \overline{\overline{G}}_{i, j}^{k+1} = -u_{i, j}^{k + \frac{1}{2}} - \frac{\tau}{2} (\Lambda_1 u_{i, j}^{k+\frac{1}{2}} + f(x_i, y_j)), \; i = 1, \ldots, N - 1,
  \end{equation}
  \begin{equation}
  \begin{gathered}
    h_x = h_y = h, \; \overline{A}_{i, j} = \overline{\overline{A}}_{i, j} = \frac{\tau}{2 h^2}, \\
    \overline{B}_{i, j} = \overline{\overline{B}}_{i, j} = \frac{\tau}{h^2} + 1, \; \overline{C}_{i, j} = \overline{\overline{C}}_{i, j} = \frac{\tau}{2 h^2},
  \end{gathered}
  \end{equation}
  \begin{equation}
    \Lambda_1 u_{i, j} = p_{i+\frac{1}{2}, j} \frac{u_{i+1, j} - u_{i, j}}{h_x^2} - p_{i-\frac{1}{2}, j} \frac{u_{i, j} - u_{i-1, j}}{h_x^2}
  \end{equation}
  \begin{equation}
    \Lambda_2 u_{i, j} = q_{i, j+\frac{1}{2}} \frac{u_{i, j+1} - u_{i, j}}{h_y^2} - q_{i, j-\frac{1}{2}} \frac{u_{i, j} - u_{i, j-1}}{h_y^2}
  \end{equation}
  \begin{equation}
    \delta_1 = c_1 \frac{4}{h_x^2} \sin^2{\frac{\pi h_x}{2 l_x}}, \; \Delta_1 = c_2 \frac{4}{h_x^2} \cos^2{\frac{\pi h_x}{2 l_x}},
  \end{equation}
  \begin{equation}
    \delta_2 = d_1 \frac{4}{h_y^2} \sin^2{\frac{\pi h_y}{2 l_y}}, \; \Delta_2 =  + d_2 \frac{4}{h_y^2} \cos^2{\frac{\pi h_y}{2 l_y}},
  \end{equation}
  \begin{equation}
    \delta = \min(\delta_1, \delta_2), \; \Delta = \max(\Delta_1, \Delta_2), \; \tau = \frac{2}{\sqrt{\delta \Delta}}
  \end{equation}
  \begin{equation}
    m \geqslant \frac{N}{2 \pi} \ln{\frac{1}{\varepsilon}}
  \end{equation}
\end{itemize}

\subsection*{Реализация}

Алгоритм реализован на языке программирования \href{https://julialang.org/}{Julia} в виде скрипта и расположен в GitLab репозитории \href{https://gitlab.com/paveloom-g/university/s09-2021/computational-workshop}{Computational Workshop S09-2021} в папке A3. Для воспроизведения результатов следуй инструкциям в файле {\footnotesize \texttt{README.md}}. \npar

Проблема:

\begin{equation}
\begin{gathered}
  L u = -f(x, y), \\
  L u = \frac{\partial^2 u}{\partial x^2} + \frac{\partial^2 u}{\partial y^2}, \\
  0 \leq x \leq 1, \; 0 \leq y \leq 1, \\
  \left. u(x, y) \right|_{\Gamma} = \mu(x, y)
\end{gathered}
\end{equation}

Для проверки алгоритма предложено решение $ u(x, t) = x^3 y + x y^2 $.

\captionsetup{singlelinecheck=false, justification=justified}

\begin{figure}[h!]
\begin{lstlisting}[
    caption=Определение проблемы
]
# Define the problem
l_x = 1
l_y = 1
p(x, y) = 1
q(x, y) = 1
u(x, y) = x^3 * y + y^2 * x
f(x, y) = -(6 * x * y + 2 * x)

# Set the expected precision of the results
ε = 1e-4
digits = 4
\end{lstlisting}
\end{figure}

\newpage

\begin{figure}[h!]
\begin{lstlisting}[
  caption={Дифференциальные операторы}
]
"Compute the value of the differential operator L"
function L(U, x, y, h_x, h_y, i, j)::Float64
    return Λ₁(U, x, y, h_x, i, j) + Λ₂(U, x, y, h_y, i, j)
end

"Compute the value of the differential operator Λ₁"
function Λ₁(U, x, y, h_x, i, j)::Float64
    return p(x[i] + h_x / 2, y[j]) *
           (U[i+1, j] - U[i, j]) / h_x^2 -
           p(x[i] - h_x / 2, y[j]) *
           (U[i, j] - U[i-1, j]) / h_x^2
end

"Compute the value of the differential operator Λ₂"
function Λ₂(U, x, y, h_y, i, j)::Float64
    return q(x[i], y[j] + h_y / 2) *
           (U[i, j+1] - U[i, j]) / h_y^2 -
           q(x[i], y[j] - h_y / 2) *
           (U[i, j] - U[i, j-1]) / h_y^2
end
\end{lstlisting}
\end{figure}

\begin{figure}[h!]
\begin{lstlisting}[
  caption={Вычисление $ \xi $ для первых трех методов}
]
"Compute ξ, which is needed to compute
 an optimal number of iterations"
function compute_xi(l_x, l_y, h_x, h_y)::Float64
    rx = 0:0.001:l_x
    ry = 0:0.001:l_y
    pairs = [(x, y) for x in rx, y in ry]
    c₁ = minimum(args -> p(args...), pairs)
    c₂ = maximum(args -> p(args...), pairs)
    d₁ = minimum(args -> q(args...), pairs)
    d₂ = maximum(args -> q(args...), pairs)
    k₁ = 4 / h_x^2
    k₂ = 4 / h_y^2
    arg₁ = (π * h_x) / (2 * l_x)
    arg₂ = (π * h_y) / (2 * l_y)
    δ = c₁ * k₁ * sin(arg₁)^2 +
        d₁ * k₂ * sin(arg₂)^2
    Δ = c₂ * k₁ * cos(arg₁)^2 +
        d₂ * k₂ * cos(arg₂)^2
    return δ / Δ
end
\end{lstlisting}
\end{figure}

\newpage

\begin{figure}[h!]
\begin{lstlisting}[
  caption={Вычисление матрицы $ U^0 $}
]
"Initialize a new matrix and compute the boundary values"
function boundary_values(N, M, x, y)::Matrix{Float64}
    # Prepare a matrix for the solution
    U = zeros(N + 1, M + 1)
    # Compute the boundary values
    U[:, 1] .= u.(x, 0)
    U[:, M+1] .= u.(x, l_y)
    U[1, 2:M] .= u.(0, y[2:M])
    U[N+1, 2:M] .= u.(l_x, y[2:M])
    return U
end
\end{lstlisting}
\end{figure}

\begin{figure}[h!]
\begin{lstlisting}[
  caption={Метод простых итераций}
]
# Compute the optimal number of iterations
m = ceil(Int, log(1 / ε) / (2ξ))
# Iterate enough times to achieve desired precision
for _ in 1:m
    # Compute values at the inner nodes
    for i = 2:N, j = 2:M
        k₁ = p(x[i] - h_x / 2, y[j]) / h_x^2
        k₂ = p(x[i] + h_x / 2, y[j]) / h_x^2
        k₃ = q(x[i], y[j] - h_y / 2) / h_y^2
        k₄ = q(x[i], y[j] + h_y / 2) / h_y^2
        Uₖ[i, j] = (k₁ * Uₖ₋₁[i-1, j] +
                    k₂ * Uₖ₋₁[i+1, j] +
                    k₃ * Uₖ₋₁[i, j-1] +
                    k₄ * Uₖ₋₁[i, j+1] +
                    f(x[i], y[j])) /
                   (k₁ + k₂ + k₃ + k₄)
    end
    # Reassign the current iteration as the previous one
    Uₖ₋₁ = Uₖ
end
# <...>
\end{lstlisting}
\end{figure}

\newpage

\begin{figure}[h!]
\begin{lstlisting}[
  caption={Метод Зейделя}
]
# Compute the optimal number of iterations
m = ceil(Int, log(1 / ε) / (4ξ))
# Iterate enough times to achieve desired precision
for _ in 1:m
    # Compute values at the inner nodes
    for i = N:-1:2, j = M:-1:2
        k₁ = p(x[i] - h_x / 2, y[j]) / h_x^2
        k₂ = p(x[i] + h_x / 2, y[j]) / h_x^2
        k₃ = q(x[i], y[j] - h_y / 2) / h_y^2
        k₄ = q(x[i], y[j] + h_y / 2) / h_y^2
        Uₖ[i, j] = (k₁ * Uₖ[i-1, j] +
                    k₂ * Uₖ₋₁[i+1, j] +
                    k₃ * Uₖ[i, j-1] +
                    k₄ * Uₖ₋₁[i, j+1] +
                    f(x[i], y[j])) /
                    (k₁ + k₂ + k₃ + k₄)
    end
    # Reassign the current iteration as the previous one
    Uₖ₋₁ = Uₖ
end
# <...>
\end{lstlisting}
\end{figure}

\newpage

\begin{figure}[h]
\begin{lstlisting}[
  caption={Метод верхней релаксации}
]
# <...>
# Compute the optimal number of iterations
m = ceil(Int, log(1 / ε) / sqrt(ξ))
# Define ω in (0,2), which affects the speed of convergence
ω = 1.0
# Iterate enough times to achieve desired precision
for _ in 1:m
    # Compute values at the inner nodes
    for i = N:-1:2, j = M:-1:2
        k₁ = p(x[i] + h_x / 2, y[j]) / h_x^2
        k₂ = p(x[i] - h_x / 2, y[j]) / h_x^2
        k₃ = q(x[i], y[j] + h_y / 2) / h_y^2
        k₄ = q(x[i], y[j] - h_y / 2) / h_y^2
        Uₖ[i, j] = Uₖ₋₁[i, j] +
                   ω * (k₁ * (Uₖ₋₁[i+1, j] - Uₖ₋₁[i, j]) -
                        k₂ * (Uₖ₋₁[i, j] - Uₖ[i-1, j]) +
                        k₃ * (Uₖ₋₁[i, j+1] - Uₖ₋₁[i, j]) -
                        k₄ * (Uₖ₋₁[i, j] - Uₖ[i, j-1]) +
                        f(x[i], y[j])) /
                   (k₁ + k₂ + k₃ + k₄)
    end
    # Reassign the current iteration as the previous one
    Uₖ₋₁ = Uₖ
end
# <...>
\end{lstlisting}
\end{figure}

\newpage

\begin{figure}[h]
\begin{lstlisting}[
  caption={Попеременно-треугольный итерационный метод (часть 1)}
]
# <...>
# Compute the coefficients
pairs = [(x, y) for x in 0:0.001:l_x, y in 0:0.001:l_y]
c₁ = minimum(args -> p(args...), pairs)
c₂ = maximum(args -> p(args...), pairs)
d₁ = minimum(args -> q(args...), pairs)
d₂ = maximum(args -> q(args...), pairs)
k₁ = 4 / h_x^2
k₂ = 4 / h_y^2
δ = c₁ * k₁ * sin((π * h_x) / (2 * l_x))^2 +
    d₁ * k₂ * sin((π * h_y) / (2 * l_y))^2
Δ = c₂ * k₁ + d₂ * k₂
ω = 2 / sqrt(δ * Δ)
η = δ / Δ
γ₁ = δ / (2 + 2 * sqrt(η))
γ₂ = δ / (4 * sqrt(η))
ξ = γ₁ / γ₂
κ₁ = ω / h_x^2
κ₂ = ω / h_y^2
τ = 2 / (γ₁ + γ₂)
# <...>
\end{lstlisting}
\end{figure}

\newpage

\begin{figure}[h]
\begin{lstlisting}[
  caption={Попеременно-треугольный итерационный метод (часть 2)}
]
# <...>
# Compute the optimal number of iterations
m = ceil(Int, log(1 / ε) / log((1 + ξ) / (1 - ξ)))
# Iterate enough times to achieve desired precision
for _ in 1:m
    # Prepare intermediate matrices
    Ũ = zeros(size(Uₖ₋₁)...)
    U̅ = zeros(size(Uₖ₋₁)...)
    # Compute values at the inner nodes of the first intermediate matrix
    for i = 2:N, j = 2:M
        k₁ = κ₁ * p(x[i] - h_x / 2, y[j])
        k₂ = κ₂ * q(x[i], y[j] - h_y / 2)
        Ũ[i, j] = (k₁ * Ũ[i-1, j] +
                   k₂ * Ũ[i, j-1] +
                   L(Uₖ₋₁, x, y, h_x, h_y, i, j) +
                   f(x[i], y[j])) /
                  (1 + k₁ + k₂)
    end
    # Compute values at the inner nodes of the second intermediate matrix
    for i = N:-1:2, j = M:-1:2
        k₁ = κ₁ * p(x[i] + h_x / 2, y[j])
        k₂ = κ₂ * q(x[i], y[j] + h_y / 2)
        U̅[i, j] = (k₁ * U̅[i+1, j] +
                   k₂ * U̅[i, j+1] + Ũ[i, j]) /
                  (1 + k₁ + k₂)
    end
    # Add up the matrices
    Uₖ[2:N, 2:M] .= Uₖ₋₁[2:N, 2:M] .+ τ .* U̅[2:N, 2:M]
    # Reassign the current iteration as the previous one
    Uₖ₋₁ = Uₖ
end
# <...>
\end{lstlisting}
\end{figure}

\newpage

\begin{figure}[h]
\begin{lstlisting}[
  caption={Метод переменных направлений (часть 1)}
]
# <...>
# Compute the optimal number of iterations
m = ceil(Int, N / (2 * π) * log(1 / ε))
# Compute the coefficients
pairs = [(x, y) for x in 0:0.001:l_x, y in 0:0.001:l_y]
c₁ = minimum(args -> p(args...), pairs)
c₂ = maximum(args -> p(args...), pairs)
d₁ = minimum(args -> q(args...), pairs)
d₂ = maximum(args -> q(args...), pairs)
k₁ = 4 / h_x^2
k₂ = 4 / h_y^2
arg₁ = (π * h_x) / (2 * l_x)
arg₂ = (π * h_y) / (2 * l_y)
δ₁ = c₁ * k₁ * sin(arg₁)^2
δ₂ = d₁ * k₂ * sin(arg₂)^2
Δ₁ = c₂ * k₁ * cos(arg₁)^2
Δ₂ = d₂ * k₁ * cos(arg₂)^2
δ = min(δ₁, δ₂)
Δ = max(Δ₁, Δ₂)
τ = 2 / sqrt(δ * Δ)
# <...>
\end{lstlisting}
\end{figure}

\newpage

\begin{figure}[h]
\begin{lstlisting}[
  caption={Метод переменных направлений (часть 2)}
]
# <...>
# Iterate enough times to achieve desired precision
for _ in 1:m
    # Compute values at the inner nodes of the intermediate matrix
    for j = 2:M
        # Compute the linear system's matrix
        Ũ = Tridiagonal(
            # A's
            [repeat([τ / (2 * h_x^2)], N - 1); 0],
            # B's
            [1; repeat([-τ / h_x^2 + 1], N - 1); 1],
            # C's
            [0; repeat([τ / (2 * h_x^2)], N - 1)],
        )
        # Compute the linear system's right-hand side vector
        g = [
            Uₖ₋₁[1, j]
            [-Uₖ₋₁[i, j] -
             τ / 2 * (Λ₂(Uₖ₋₁, x, y, h_y, i, j) +
                      f(x[i], y[j]))
             for i in 2:N
            ]
            Uₖ₋₁[N+1, j]]
        # Compute the solution of the linear system
        U̅[:, j] .= Ũ \ g
    end
# <...>
\end{lstlisting}
\end{figure}

\newpage

\begin{figure}[h]
\begin{lstlisting}[
  caption={Метод переменных направлений (часть 3)}
]
# <...>
    # Compute values at the inner nodes of the input matrix
    for i = 2:N
        # Compute the linear system's matrix
        Ũ = Tridiagonal(
            # A's
            [repeat([τ / (2 * h_y^2)], N - 1); 0],
            # B's
            [1; repeat([-τ / h_y^2 + 1], N - 1); 1],
            # C's
            [0; repeat([τ / (2 * h_y^2)], N - 1)],
        )
        # Compute the linear system's
        # right-hand side vector
        g = [
            Uₖ₋₁[i, 1]
            [-U̅[i, j] -
             τ / 2 * (Λ₁(U̅, x, y, h_x, i, j) +
                      f(x[i], y[j]))
             for j in 2:M
            ]
            Uₖ₋₁[i, M+1]]
        # Compute the solution of the linear system
        Uₖ[i, :] .= Ũ \ g
    end
    # Reassign the current iteration as the previous one
    Uₖ₋₁ = Uₖ
end
# <...>
\end{lstlisting}
\end{figure}

\newpage

\captionsetup{justification=centering}

\begin{table}[h]
  \centering
  \caption{Решение дифференциального уравнения}
  \renewcommand{\arraystretch}{1.2}
  \begin{tabular}{!{\vrule width 1pt}c!{\vrule width 0.5pt}cccccc}
    \specialrule{\heavyrulewidth}{0pt}{0pt}
    \diagbox[height=2.0\line]{$ x $}{$ y $} &
    $ 0 $ &
    $ 0.2 $ &
    $ 0.4 $ &
    $ 0.6 $ &
    $ 0.8 $ &
    $ 1 $ \\
    \specialrule{\lightrulewidth}{0pt}{0pt}
    \arrayrulecolor{black!40}
    $ 0 $ & $ 0.0000 $ & $ 0.0000 $ & $ 0.0000 $ & $ 0.0000 $ & $ 0.0000 $ & $ 0.0000 $ \\\cline{2-7}
    $ 0.2 $ & $ 0.0000 $ & $ 0.0096 $ & $ 0.0352 $ & $ 0.0768 $ & $ 0.1344 $ & $ 0.2080 $ \\\cline{2-7}
    $ 0.4 $ & $ 0.0000 $ & $ 0.0288 $ & $ 0.0896 $ & $ 0.1824 $ & $ 0.3072 $ & $ 0.4640 $ \\\cline{2-7}
    $ 0.6 $ & $ 0.0000 $ & $ 0.0672 $ & $ 0.1824 $ & $ 0.3456 $ & $ 0.5568 $ & $ 0.8160 $ \\\cline{2-7}
    $ 0.8 $ & $ 0.0000 $ & $ 0.1344 $ & $ 0.3328 $ & $ 0.5952 $ & $ 0.9216 $ & $ 1.3120 $ \\\cline{2-7}
    $ 1.0 $ & $ 0.0000 $ & $ 0.2400 $ & $ 0.5600 $ & $ 0.9600 $ & $ 1.4400 $ & $ 2.0000 $ \\\cline{2-7}
    \arrayrulecolor{black}
    \specialrule{\heavyrulewidth}{0pt}{0pt}
  \end{tabular}
\end{table}

\begin{table}[h]
  \centering
  \caption{Сравнение результатов с точным решением}
  \begin{tabular}{ccccccc}
    \toprule
    $ N, M $ &
    $ m_1, \Delta u_1 $ &
    $ m_2, \Delta u_2 $ &
    $ m_3, \Delta u_3 $ &
    $ m_4, \Delta u_4 $ &
    $ m_5, \Delta u_5 $ \\
    \midrule
    $ 5 $ & $ 44, 10^{-9} $ & $ 22, 10^{-13} $ & $ 29, 10^{-16} $ & $ 9, 10^{-16} $ & $ 8, 10^{-15} $ \\
    \arrayrulecolor{black!40}
    \midrule
    $ 10 $ & $ 184, 10^{-9} $ & $ 92, 10^{-13} $ & $ 59, 10^{-15} $ & $ 17, 10^{-16} $ & $ 15, 10^{-14} $ \\
    \midrule
    $ 20 $ & $ 744, 10^{-9} $ & $ 372, 10^{-13} $ & $ 118, 10^{-14} $ & $ 32, 10^{-16} $ & $ 30, 10^{-14} $ \\
    \arrayrulecolor{black}
    \bottomrule
  \end{tabular}
\end{table}

\end{document}
