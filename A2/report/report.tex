\documentclass[a4paper, oneside]{article}
\special{pdf:minorversion 6}

\usepackage[english, russian]{babel}

\usepackage{fontspec}
\setmainfont[
  Ligatures=TeX,
  Extension=.otf,
  BoldFont=cmunbx,
  ItalicFont=cmunti,
  BoldItalicFont=cmunbi,
]{cmunrm}
\usepackage{unicode-math}

\usepackage[bookmarks=false]{hyperref}
\hypersetup{pdfstartview={FitH},
            colorlinks=true,
            linkcolor=magenta,
            pdfauthor={Павел Соболев}}

\usepackage{calrsfs}
\DeclareMathAlphabet{\pazocal}{OMS}{zplm}{m}{n}

\usepackage[table]{xcolor}
\usepackage{booktabs}
\usepackage{caption}

\usepackage{graphicx}
\graphicspath{ {../plots/} }

\usepackage{sectsty}
\sectionfont{\centering}
\subsubsectionfont{\centering\normalfont\itshape}

\newcommand{\npar}{\par\vspace{\baselineskip}}

\usepackage{float}
\input{../../shared/julia_font}
\input{../../shared/julia_listings}
\lstset{language=Julia, style=julia}

\setlength{\parindent}{0pt}


\hypersetup{pdftitle={Вычислительный практикум (9-ый семестр, 2021)}}

\begin{document}

\subsubsection*{Вычислительный практикум (9-ый семестр, 2021)}
\section*{Дифференциальные уравнения параболического типа}
\subsubsection*{Руководитель: А. Г. Доронина \hspace{2em} Выполнил: П. Л. Соболев}

\vspace{3em}

\subsection*{Задачи}

\begin{itemize}
  \setlength\itemsep{-0.1em}
  \item Создать алгоритм решения дифференциального уравнения параболического типа при заданных видах дифференциального оператора и граничных условий и протестировать его на выбранном решении.
\end{itemize}

\subsection*{Теория}

Постановка задачи: найти решение $ u(x, t) $ проблемы

\su
\begin{equation}
  \frac{\partial u}{\partial t} = L u + f(x, t), \; 0 < x < 1, \; 0 < t \leqslant T,
\end{equation}

\su\su
\begin{equation}
  u(x, 0) = \varphi(x), \; 0 \leqslant x \leqslant 1,
\end{equation}

\su\su
\begin{equation}
  \left. \left( \alpha_1(t) u - \alpha_2(t) \frac{\partial u}{\partial x} \right) \right|_{x=0} = \alpha(t), \; 0 \leqslant t \leqslant 0.1,
\end{equation}

\su
\begin{equation}
\left. \left( \beta_1(t) u + \beta_2(t) \frac{\partial u}{\partial x} \right) \right|_{x=0} = \beta(t), \; 0 \leqslant t \leqslant 0.1,
\end{equation}

где

\su
\begin{equation}
  Lu = \begin{sqcases}
    a(x, t) \dfrac{\partial^2 u}{\partial x^2} + b(x,t) \dfrac{\partial u}{\partial x} + c(x, t) u; \\
    \dfrac{\partial}{\partial x} \left( p(x) \dfrac{\partial u}{\partial x} \right) + b(x, t) \dfrac{\partial u}{\partial x} + c(x, t) u.
  \end{sqcases}
\end{equation}

Разностный аналог дифференциального оператора (5) на равномерной сетке $ \left\{ (x_i = h i, t_k = \tau k), \; h = 1 / N, \; \tau = 1 / M, \; i = \overline{0, N}, \; k = \overline{0, M} \right\} $:

\su\su
\begin{equation}
  L_h u_i^k = \begin{sqcases}
    a(x_i, t_k) \frac{u_{i+1}^k - 2 u_i^k + u_{i-1}^k}{h^2} + b(x_i, t_k) \frac{u_{i+1}^k - u_{i-1}^k}{2h} + c(x_i, t_k) u_i^k; \\
    p_{i+\frac{1}{2}} \frac{u_{i+1}^k - u_i^k}{h^2} - p_{i - \frac{1}{2}} \frac{u_i^k - u_i^k}{h^2} + b(x_i, t_k) \frac{u_{i+1}^k - u_{i-1}^k}{2h} + c(x_i, t_k) u_i^k.
  \end{sqcases}
\end{equation}

Рассмотрим однопараметрическое семейство разностных схем:

\su
\begin{equation}
\begin{gathered}
  \frac{u_i^k - u_i^{k-1}}{\tau} = L_h(\sigma u_i^k + (1 - \sigma) u_i^{k-1}) + f(x, \overline{t_k}), \\
  \sigma \in [0, 1], \; \overline{t_k} = t_k + (1 - \sigma) \tau, \; i = \overline{1, N-1}, \; k = \overline{1, M}.
\end{gathered}
\end{equation}

Из начального условия (2) имеем

\su
\begin{equation}
  u_i^0 = \varphi(x_i), \; i = \overline{0, N}.
\end{equation}

Если $ \sigma = 0 $, то разностная схема (7) является явной. Аппроксимируя производные в граничных условиях (3)--(4) с порядком $ O(h^2) $, получаем следующие выражения для вычисления значений решения при $ k = \overline{1, M} $:

\su
\begin{equation}
  u_i = u_i^{k-1} + \tau (L_h u_i^{k-1} + f(x_i, t_{k-1})), \; i = \overline{1, N-1},
\end{equation}

\su
\begin{equation}
  \alpha_1(t_k) u_0^k - \alpha_2(t_k) \frac{-3 u_0^k + 4 u_1^k - u_2^k}{2h} = \alpha(t_k),
\end{equation}

\su
\begin{equation}
  \beta_1(t_k) u_N^k + \beta_2(t_k) \frac{3 u_N^k - 4 u_{N-1}^k + u_{N-2}^k}{2h} = \beta(t_k).
\end{equation}

Условие устойчивости явной разностной схемы:

\su
\begin{equation}
  A \nu \leqslant \frac{1}{2},
\end{equation}

где $ \nu = \tau / h^2 $, а

\su
\begin{equation}
  A = \begin{sqcases}
    \text{max} \left\{ a(x, t) \; | \; 0 \leqslant x \leqslant 1, \; 0 \leqslant t \leqslant T \right\}; \\
    \text{max} \left\{ p(x) \; | \; 0 \leqslant x \leqslant 1 \right\}.
  \end{sqcases}
\end{equation}

Если $ \sigma \neq 0 $, то разностная схема (7) является неявной и может быть представлена в виде

\su
\begin{equation}
  \sigma L_h u_i^k - \frac{1}{\tau} u_i^k = G_i^k,
\end{equation}

где

\su\su\su
\begin{equation}
  G_i^k = -\frac{1}{\tau} u_i^{k-1} - (1 - \sigma) L_h u_i^{k-1} - f(x_i, \overline{t_k}), \; i = \overline{1, N-1}, \; k = \overline{1, M}.
\end{equation}

Аппроксимируя производные в граничных условиях (3)--(4) с порядком \\ $ O(h) $, получаем

\su\su
\begin{equation}
  \alpha_1(t_k) u_0^k - \alpha_2(t_k) \frac{u_1^k - u_0^k}{h} = \alpha(t_k),
\end{equation}

\su
\begin{equation}
  \beta_1(t_k) u_N^k + \beta_2(t_k) \frac{u_N^k - u_{N-1}^k}{h} = \beta(t_k).
\end{equation}

Приведя граничные условия (16)--(17) к виду

\su
\begin{equation}
\begin{gathered}
  - B_0 u_0^k + C_0 u_1^k = G_0^k, \\
  A_N u_{N-1}^k - B_N u_N^k = G_N^k
\end{gathered}
\end{equation}

и объединив их в систему с выражениями (14)--(15), получаем следующую систему линейных уравнений для вычисления значений решения при $ k = \overline{1, M} $:

\begin{equation}
\begin{alignedat}{4}
  & -B_0 u_0^k && + C_0 u_1^k && = G_0^k, \\
  A_i u_{i-1}^k & - B_i u_i^k && + C_i u_{i+1}^k && = G_i^k, \; i = \overline{1, N-1}, \\
  A_N u_{N-1}^k & - B_N u_N^k && && = G_N^k.
\end{alignedat}
\end{equation}

\subsection*{Реализация}

Алгоритм реализован на языке программирования \href{https://julialang.org/}{Julia} в виде скрипта и расположен в GitLab репозитории \href{https://gitlab.com/paveloom-g/university/s09-2021/computational-workshop}{Computational Workshop S09-2021} в папке A2. Для воспроизведения результатов следуй инструкциям в файле {\footnotesize \texttt{README.md}}. \npar

\newpage

Проблема:

\begin{equation}
\begin{gathered}
  \frac{\partial u}{\partial t} = \frac{\partial}{\partial x} \left( (x + 3) \frac{\partial u}{\partial x} \right) - xu + f(x, t), \\
  u(x, 0) = \varphi(x), \; 0 \leqslant x \leqslant 1, \\
  \left. \frac{\partial u}{\partial x} \right|_{x=0} = \alpha(t), \; u(1, t) = \beta(t), \; 0 \leqslant t \leqslant 0.1.
\end{gathered}
\end{equation}

Для проверки алгоритма возьмем решение $ u(x, t) = x + t $.

\captionsetup{singlelinecheck=false, justification=justified}

\begin{figure}[h!]
\begin{lstlisting}[
    caption=Определение функций дифф. уравнения и проблемы
]
# Define the differential equation's functions
p(x) = x + 3
b(_, _) = 0
c(x, _) = -x
α₁(t) = 0
α₂(t) = -1
β₁(t) = 1
β₂(t) = 0

# Define the problem's functions
u(x, t) = x + 3t
f(x, t) = x^2 + 3 * x * t + 2
φ(x) = x
α(t) = 1
β(t) = 1 + 3t
\end{lstlisting}
\end{figure}

\begin{figure}[h!]
\begin{lstlisting}[
  caption={Поиск сетки, дающей стабильное решение для явной схемы}
]
"Check if a solution will be stable on the specified grid"
function is_stable(N, M)::Bool
    A = maximum(p, 0:0.0001:1)
    h = 1 / N
    τ = T / M
    ν = τ / h^2
    return A * ν ≤ 0.5
end

"Using provided N, find M, which gives a stable solution"
function find_stable(N)::Int
    M = 5
    while !is_stable(N, M)
        M *= 2
    end
    return M
end
\end{lstlisting}
\end{figure}

\newpage

\begin{figure}[h!]
\begin{lstlisting}[
  caption={Вычисление дифференциального оператора $ L $}
]
"Calculate the value of the differential operator L"
function L(um, x, t, h, i, k)::Float64
    return p(x[i] + h/2) * (um[k,i+1] - um[k,i]) / h^2 -
           p(x[i] - h/2) * (um[k,i] - um[k,i-1]) / h^2 +
           b(x[i], t[k]) * (um[k,i+1] - um[k,i-1]) / (2h) +
           c(x[i], t[k]) * um[k, i]
end
\end{lstlisting}
\end{figure}

\begin{figure}[h!]
\begin{lstlisting}[
  caption={Вычисление решения по явной схеме}
]
# <...>
# Prepare a matrix for the solution
# (number of columns is the number of nodes for x's)
um = Matrix{Float64}(undef, M + 1, N + 1)
# Compute the first layer
um[1, :] .= φ.(x)
# Use the explicit scheme for other layers
for k = 2:M+1
    # Compute for i in 2:N
    for i = 2:N
        um[k, i] = um[k-1,i] + τ * f(x[i],t[k-1]) +
        τ * L(um, x, t, h, i, k - 1)
    end
    # Compute the rest
    um[k, 1] = (α(t[k]) + α₂(t[k]) *
               (4 * um[k, 2] - um[k, 3]) / (2h)) /
               (α₁(t[k]) + 3 * α₂(t[k]) / (2h))
    um[k, N+1] = (β(t[k]) + β₂(t[k]) *
                 (4 * um[k, N] - um[k, N-1]) / (2h)) /
                 (β₁(t[k]) + 3 * β₂(t[k]) / (2h))
end
# <...>
\end{lstlisting}
\end{figure}

\newpage

\begin{figure}[h!]
\begin{lstlisting}[
  caption={Вычисление решения по неявной схеме}
]
# <...>
# Prepare a matrix for the solution
# (number of columns is the number of nodes for x's)
um = Matrix{Float64}(undef, M + 1, N + 1)
# Compute the first layer
um[1, :] .= φ.(x)
# Use the implicit scheme for other layers
for k = 2:M+1
    # Compute the linear system's matrix
    tm = Tridiagonal(
        # A's
        [
            [
                σ * (p(x[i] - h / 2) / h^2 -
                b(x[i], t[k]) / (2h))
                for i = 2:N
            ]; -β₂(t[k]) / h
        ],
        # B's
        [
            α₁(t[k]) + α₂(t[k]) / h; -[
                1 / τ + σ * ((p(x[i] + h / 2) +
                p(x[i] - h / 2)) / h^2 -
                c(x[i], t[i]))
                for i = 2:N
            ]; β₁(t[k]) + β₂(t[k]) / h
        ],
        # C's
        [
            -α₂(t[k]) / h; [
              σ * (p(x[i] + h / 2) / h^2 +
              b(x[i], t[k]) / (2h))
              for i = 2:N
            ]
        ],
    )
    # Compute the linear system's right-hand side vector
    g = [
        α(t[k]); [
          -um[k-1, i] / τ - (1 - σ) *
          L(um, x, t, h, i, k - 1) -
          f(x[i], t[k] - (1 - σ) * τ)
          for i = 2:N
        ]; β(t[k])
    ]
    # Compute the solution of the linear system
    um[k, :] = tm \ g
end
\end{lstlisting}
\end{figure}

\newpage

\captionsetup{justification=centering}

\begin{table}[h]
  \centering
  \caption{Решение дифференциального уравнения}
  \renewcommand{\arraystretch}{1.2}
  \begin{tabular}{!{\vrule width 1pt}c!{\vrule width 0.5pt}cccccc}
    \specialrule{\heavyrulewidth}{0pt}{0pt}
    \diagbox[height=2.0\line]{$ t $}{$ x $} &
    $ 0 $ &
    $ 0.2 $ &
    $ 0.4 $ &
    $ 0.6 $ &
    $ 0.8 $ &
    $ 1 $ \\
    \specialrule{\lightrulewidth}{0pt}{0pt}
    \arrayrulecolor{black!40}
    $ 0 $ & $ 0.0 $ & $ 0.2 $ & $ 0.4 $ & $ 0.6 $ & $ 0.8 $ & $ 1.0 $ \\\cline{2-7}
    $ 0.02 $ & $ 0.06 $ & $ 0.26 $ & $ 0.46 $ & $ 0.66 $ & $ 0.86 $ & $ 1.06 $ \\\cline{2-7}
    $ 0.04 $ & $ 0.12 $ & $ 0.32 $ & $ 0.52 $ & $ 0.72 $ & $ 0.92 $ & $ 1.12 $ \\\cline{2-7}
    $ 0.06 $ & $ 0.18 $ & $ 0.38 $ & $ 0.58 $ & $ 0.78 $ & $ 0.98 $ & $ 1.18 $ \\\cline{2-7}
    $ 0.08 $ & $ 0.24 $ & $ 0.44 $ & $ 0.64 $ & $ 0.84 $ & $ 1.04 $ & $ 1.24 $ \\\cline{2-7}
    $ 0.1 $ & $ 0.3 $ & $ 0.5 $ & $ 0.7 $ & $ 0.9 $ & $ 1.1 $ & $ 1.3 $ \\\cline{2-7}
    \arrayrulecolor{black}
    \specialrule{\heavyrulewidth}{0pt}{0pt}
  \end{tabular}
\end{table}

\begin{table}[h]
  \centering
  \caption{Сравнение результата с точным решением на явной схеме}
  \begin{tabular}{ccc}
    \toprule
    $ h $ &
    $ \tau $ &
    $ \Delta u $ \\
    \midrule
    $ 0.2 $ & $ 0.05 $ & $ 2.220446049250313 \cdot 10^{-16} $ \\
    \arrayrulecolor{black!40}
    \midrule
    $ 0.1 $ & $ 0.0125 $ & $ 3.608224830031759 \cdot 10^{-16} $ \\
    \midrule
    $ 0.05 $ & $ 0.003125 $ & $ 5.551115123125783 \cdot 10^{-16} $ \\
    \arrayrulecolor{black}
    \bottomrule
  \end{tabular}
\end{table}

\begin{table}[h]
  \centering
  \caption{Сравнение результата с точным решением на неявной схеме}
  \begin{tabular}{cccc}
    \toprule
    $ h $ &
    $ \tau $ &
    $ \Delta u \; (\sigma = 0.5) $ &
    $ \Delta u \; (\sigma = 1.0) $ \\
    \midrule
    $ 0.2 $ & $ 0.001 $ & $ 5.639932965095795 \cdot 10^{-14} $ & $ 2.403632848313464 \cdot 10^{-14} $ \\
    \arrayrulecolor{black!40}
    \midrule
    $ 0.1 $ & $ 0.001 $ & $ 1.343369859796439 \cdot 10^{-14} $ & $ 3.758104938356155 \cdot 10^{-14} $ \\
    \midrule
    $ 0.05 $ & $ 0.001 $ & $ 6.361577931102147 \cdot 10^{-14} $ & $ 2.675637489346627 \cdot 10^{-14} $ \\
    \arrayrulecolor{black}
    \bottomrule
  \end{tabular}
\end{table}

\end{document}
